\documentclass{amsproc}



%%%%%%%%%%%%    Package Section    %%%%%%%%%%%%

 \usepackage{amssymb}
     % For the usage of some mathematical symbols

% \usepackage{graphicx}
     % For the ability to put in pictures into latex files

 \usepackage{amsmath}
     % For the ability to do other forms of math things

 \usepackage{amstext}
     % For the ability to type in ams style, I guess?
     
% \usepackage{mhequ}
     % For ability to have multiple columns on single line
     % Automatically puts into math mode, so could also just use in place of regular equation enviorment
     % Environment called equation
     % \begin{equation}[ (number of columns wanted) ]
     % example
         % \begin{equation}[3]
             % x &= y +2  &\qquad y &= 5   &\qquad  z &= 4
    % \end{equation}
    
    %Splits up into 3 columns based on \qquad, centers individual columns with & and centers division of columns with &
    % Can also use \multicol{ ( Number of columns desired to be smashed together ) } { Text to go into larger combined column }

 \usepackage{esvect}
      % For ability to use better vector notation
      % $ \vv{o} ^ t $
      
\usepackage{commath}
    % For ability to use norm and abs functions 
 
%\usepackage{lmodern}
%\usepackage{scrextend}
%\changefontsizes[16pt]{12pt}
    % For changing the default text size

\usepackage{algpseudocode}
    % For ability to write pseudocode. 
    % \If{condition} <state> \ElsIf{condition} <state> \Else <state> \EndIf
        % Note: Only \If and \Endif statements necessary.
    % \For{condition} <state> \EndFor
    % \ForAll{condition} <state> \EndFor
    % \While{condition} <state> \EndWhile
    % All of this can only be done within an \begin{algorithmic} block.

%%%%%%%%%%%%%%%%%%%%%%%%%
%%%%%%%%%%%%%%%%%%%%%%%%%
%%%%%%%%%%%%%%%%%%%%%%%%%
%%%%%%%%%%%%%%%%%%%%%%%%%
%%%%%%%%%%%%%%%%%%%%%%%%%
%%%%%%%%%%%%%%%%%%%%%%%%%
%%%%%%%%%%%%%%%%%%%%%%%%%
%
% Misceleanous 
%
%%%%%%%%%%%%%%%%%%%%%%%%%
%%%%%%%%%%%%%%%%%%%%%%%%%
%%%%%%%%%%%%%%%%%%%%%%%%%
%%%%%%%%%%%%%%%%%%%%%%%%%
%%%%%%%%%%%%%%%%%%%%%%%%%
%%%%%%%%%%%%%%%%%%%%%%%%%
%%%%%%%%%%%%%%%%%%%%%%%%%


\title{Algorithm describing current setup for Distributed IVA}

%%%%%%%%%%%%%%%%%%%%%%%%%
%%%%%%%%%%%%%%%%%%%%%%%%%
%%%%%%%%%%%%%%%%%%%%%%%%%
%%%%%%%%%%%%%%%%%%%%%%%%%
%%%%%%%%%%%%%%%%%%%%%%%%%
%%%%%%%%%%%%%%%%%%%%%%%%%
%%%%%%%%%%%%%%%%%%%%%%%%%
%
% Document Start
%
%%%%%%%%%%%%%%%%%%%%%%%%%
%%%%%%%%%%%%%%%%%%%%%%%%%
%%%%%%%%%%%%%%%%%%%%%%%%%
%%%%%%%%%%%%%%%%%%%%%%%%%
%%%%%%%%%%%%%%%%%%%%%%%%%
%%%%%%%%%%%%%%%%%%%%%%%%%
%%%%%%%%%%%%%%%%%%%%%%%%%


\begin{document}

\maketitle

%%%%%%%%%%%%%%%%%%%%%%%%%
%%%%%%%%%%%%%%%%%%%%%%%%%
%%%%%%%%%%%%%%%%%%%%%%%%%
%%%%%%%%%%%%%%%%%%%%%%%%%
%%%%%%%%%%%%%%%%%%%%%%%%%
%%%%%%%%%%%%%%%%%%%%%%%%%
%%%%%%%%%%%%%%%%%%%%%%%%%
\section{Notation} Here we will describe notation as well as assumptions.
%%%%%%%%%%%%%%%%%%%%%%%%%
%%%%%%%%%%%%%%%%%%%%%%%%%
%%%%%%%%%%%%%%%%%%%%%%%%%
%%%%%%%%%%%%%%%%%%%%%%%%%
%%%%%%%%%%%%%%%%%%%%%%%%%
%%%%%%%%%%%%%%%%%%%%%%%%%
%%%%%%%%%%%%%%%%%%%%%%%%%

There are $K$ total sites

There are $N$ subjects per site

There are $C$ components per subject

$W^i_l$ denotes the unmixing matrix for subject $l$ at site $i$

$X^i_l$ denotes the $l^{th}$ subject at site $i$

$Y^i_l$ denotes the approximation to subject $l$ at site $i$

$\hat{Y}^i$ denotes the site source for site $i$. Explained further in algorithm section

$W^{Gi}$ denotes the master node unmixing matrix for the $i^{th}$ site

$S^i_l$ denotes the master node approximation to the true source of site $i$



At each site, there is a sub IVA (or ICA / group ICA) problem of finding statistically independent sources for each subject

\[
    \begin{pmatrix}
        W^i_1   & 0         & \dots     & 0         \\
        0       & W^i_2     & \dots     & 0         \\
        \vdots  & \vdots    & \ddots    & \vdots    \\
        0       & 0         & \dots     & W^i_N
    \end{pmatrix}
    %
    \begin{pmatrix}
        X^i_l   \\
        X^i_2   \\
        \vdots  \\
        X^i_N
    \end{pmatrix}
    %
    =
    %
    \begin{pmatrix}
        Y^i_1   \\
        Y^i_2   \\
        \vdots  \\
        Y^i_N
    \end{pmatrix}
\]

At the master node, there is a main IVA (or ICA / group ICA) problem of finding statistically independent sources for the site sources $\hat{Y}$
        
\[
    \begin{pmatrix}
        W^{G1}  & 0         & \dots     & 0         \\
        0       & W^{G2}    & \dots     & 0         \\
        \vdots  & \vdots    & \ddots    & \vdots    \\
        0       & 0         & \dots     & W^{GN}
    \end{pmatrix}
    %
    \begin{pmatrix}
        \hat{Y}^1   \\
        \hat{Y}^2   \\
        \vdots      \\
        \hat{Y}^N
    \end{pmatrix}
    %
    =
    %
    \begin{pmatrix}
        S^1     \\
        S^2     \\
        \vdots  \\
        S^N
    \end{pmatrix}
\]


%%%%%%%%%%%%%%%%%%%%%%%%%
%%%%%%%%%%%%%%%%%%%%%%%%%
%%%%%%%%%%%%%%%%%%%%%%%%%
%%%%%%%%%%%%%%%%%%%%%%%%%
%%%%%%%%%%%%%%%%%%%%%%%%%
%%%%%%%%%%%%%%%%%%%%%%%%%
%%%%%%%%%%%%%%%%%%%%%%%%%
\section{Algorithm} We will now describe the current algorithm
%%%%%%%%%%%%%%%%%%%%%%%%%
%%%%%%%%%%%%%%%%%%%%%%%%%
%%%%%%%%%%%%%%%%%%%%%%%%%
%%%%%%%%%%%%%%%%%%%%%%%%%
%%%%%%%%%%%%%%%%%%%%%%%%%
%%%%%%%%%%%%%%%%%%%%%%%%%
%%%%%%%%%%%%%%%%%%%%%%%%%


\begin{algorithm}
While not all sites and master node have converged 
    for all sites k = 1..K
        compute
\end{algorithm}

\end{document}