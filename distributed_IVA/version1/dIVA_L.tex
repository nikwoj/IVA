\documentclass{amsproc}



%%%%%%%%%%%%    Package Section    %%%%%%%%%%%%

 \usepackage{amssymb}
     % For the usage of some mathematical symbols

% \usepackage{graphicx}
     % For the ability to put in pictures into latex files

 \usepackage{amsmath}
     % For the ability to do other forms of math things

 \usepackage{amstext}
     % For the ability to type in ams style, I guess?
     
% \usepackage{mhequ}
     % For ability to have multiple columns on single line
     % Automatically puts into math mode, so could also just use in place of regular equation enviorment
     % Environment called equation
     % \begin{equation}[ (number of columns wanted) ]
     % example
         % \begin{equation}[3]
             % x &= y +2  &\qquad y &= 5   &\qquad  z &= 4
    % \end{equation}
    
    %Splits up into 3 columns based on \qquad, centers individual columns with & and centers division of columns with &
    % Can also use \multicol{ ( Number of columns desired to be smashed together ) } { Text to go into larger combined column }

 \usepackage{esvect}
      % For ability to use better vector notation
      % $ \vv{o} ^ t $
      
\usepackage{commath}
    % For ability to use norm and abs functions 
 
%\usepackage{lmodern}
%\usepackage{scrextend}
%\changefontsizes[16pt]{12pt}
    % For changing the default text size

\usepackage{algpseudocode}
    % For ability to write pseudocode. 
    % \If{condition} <state> \ElsIf{condition} <state> \Else <state> \EndIf
        % Note: Only \If and \Endif statements necessary.
    % \For{condition} <state> \EndFor
    % \ForAll{condition} <state> \EndFor
    % \While{condition} <state> \EndWhile
    % All of this can only be done within an \begin{algorithmic} block.

%%%%%%%%%%%%%%%%%%%%%%%%%
%%%%%%%%%%%%%%%%%%%%%%%%%
%%%%%%%%%%%%%%%%%%%%%%%%%
%%%%%%%%%%%%%%%%%%%%%%%%%
%%%%%%%%%%%%%%%%%%%%%%%%%
%%%%%%%%%%%%%%%%%%%%%%%%%
%%%%%%%%%%%%%%%%%%%%%%%%%
%
% Misceleanous 
%
%%%%%%%%%%%%%%%%%%%%%%%%%
%%%%%%%%%%%%%%%%%%%%%%%%%
%%%%%%%%%%%%%%%%%%%%%%%%%
%%%%%%%%%%%%%%%%%%%%%%%%%
%%%%%%%%%%%%%%%%%%%%%%%%%
%%%%%%%%%%%%%%%%%%%%%%%%%
%%%%%%%%%%%%%%%%%%%%%%%%%


\title{Algorithm describing current setup for Distributed IVA}

%%%%%%%%%%%%%%%%%%%%%%%%%
%%%%%%%%%%%%%%%%%%%%%%%%%
%%%%%%%%%%%%%%%%%%%%%%%%%
%%%%%%%%%%%%%%%%%%%%%%%%%
%%%%%%%%%%%%%%%%%%%%%%%%%
%%%%%%%%%%%%%%%%%%%%%%%%%
%%%%%%%%%%%%%%%%%%%%%%%%%
%
% Document Start
%
%%%%%%%%%%%%%%%%%%%%%%%%%
%%%%%%%%%%%%%%%%%%%%%%%%%
%%%%%%%%%%%%%%%%%%%%%%%%%
%%%%%%%%%%%%%%%%%%%%%%%%%
%%%%%%%%%%%%%%%%%%%%%%%%%
%%%%%%%%%%%%%%%%%%%%%%%%%
%%%%%%%%%%%%%%%%%%%%%%%%%


\begin{document}

\maketitle

%%%%%%%%%%%%%%%%%%%%%%%%%
%%%%%%%%%%%%%%%%%%%%%%%%%
%%%%%%%%%%%%%%%%%%%%%%%%%
%%%%%%%%%%%%%%%%%%%%%%%%%
%%%%%%%%%%%%%%%%%%%%%%%%%
%%%%%%%%%%%%%%%%%%%%%%%%%
%%%%%%%%%%%%%%%%%%%%%%%%%
\section{Notation} Here we will describe notation as well as assumptions.
%%%%%%%%%%%%%%%%%%%%%%%%%
%%%%%%%%%%%%%%%%%%%%%%%%%
%%%%%%%%%%%%%%%%%%%%%%%%%
%%%%%%%%%%%%%%%%%%%%%%%%%
%%%%%%%%%%%%%%%%%%%%%%%%%
%%%%%%%%%%%%%%%%%%%%%%%%%
%%%%%%%%%%%%%%%%%%%%%%%%%

There are $K$ total sites

There are $N$ subjects per site

There are $C$ components per subject

$W^k_l$ denotes the unmixing matrix for subject $l$ at site $k$

$X^k_l$ denotes the $l^{th}$ subject at site $k$

$Y^k_l$ denotes the approximation to subject $l$ at site $k$

$\hat{Y}^k$ denotes the site source for site $k$. Explained further in algorithm section

$W^{Gk}$ denotes the master node unmixing matrix for the $i^{th}$ site. Note that this starts out as the identity

$S^k_l$ denotes the master node approximation to the true source of site $k$



At each site, $k=1 \dots K$ there is a sub IVA (or ICA / group ICA) problem of finding statistically independent sources $Y^k_l$ for each subject

\begin{align} \label{site_specific}
    \begin{pmatrix}
        W^k_1   & 0         & \dots     & 0         \\
        0       & W^k_2     & \dots     & 0         \\
        \vdots  & \vdots    & \ddots    & \vdots    \\
        0       & 0         & \dots     & W^k_N
    \end{pmatrix}
    %
    \begin{pmatrix}
        X^k_l   \\
        X^k_2   \\
        \vdots  \\
        X^k_N
    \end{pmatrix}
    %
    =
    %
    \begin{pmatrix}
        Y^k_1   \\
        Y^k_2   \\
        \vdots  \\
        Y^k_N
    \end{pmatrix}
\end{align} 

At the master node, there is a main IVA (or ICA / group ICA) problem of finding statistically independent sources for the site sources $\hat{Y}$
        
\begin{align} \label{global}
    \begin{pmatrix}
        W^{G1}  & 0         & \dots     & 0         \\
        0       & W^{G2}    & \dots     & 0         \\
        \vdots  & \vdots    & \ddots    & \vdots    \\
        0       & 0         & \dots     & W^{GK}
    \end{pmatrix}
    %
    \begin{pmatrix}
        \hat{Y}^1   \\
        \hat{Y}^2   \\
        \vdots      \\
        \hat{Y}^K
    \end{pmatrix}
    %
    =
    %
    \begin{pmatrix}
        S^1     \\
        S^2     \\
        \vdots  \\
        S^K
    \end{pmatrix}
\end{align}

\pagebreak

%%%%%%%%%%%%%%%%%%%%%%%%%
%%%%%%%%%%%%%%%%%%%%%%%%%
%%%%%%%%%%%%%%%%%%%%%%%%%
%%%%%%%%%%%%%%%%%%%%%%%%%
%%%%%%%%%%%%%%%%%%%%%%%%%
%%%%%%%%%%%%%%%%%%%%%%%%%
%%%%%%%%%%%%%%%%%%%%%%%%%
\section{Algorithm} We will now describe the current algorithm
%%%%%%%%%%%%%%%%%%%%%%%%%
%%%%%%%%%%%%%%%%%%%%%%%%%
%%%%%%%%%%%%%%%%%%%%%%%%%
%%%%%%%%%%%%%%%%%%%%%%%%%
%%%%%%%%%%%%%%%%%%%%%%%%%
%%%%%%%%%%%%%%%%%%%%%%%%%
%%%%%%%%%%%%%%%%%%%%%%%%%


\begin{algorithmic}
 
\For {all sites k = 1..K}
    \State Site k does its own IVA to get source vetors for each subject
    \State Create $\hat{Y}^k$ out of components of $Y^k_l$
    \State Send the $\hat{Y}^k$ to the master node
\EndFor
\State Master node computes Global IVA, as seen in \ref{global}
\State Master node sends $W^{Gk}$ back to site k
\For {all sites k = 1..K}
    \State Site k recomputes unmixing matrices by $W^k_i = W^{Gk} * W^k_i$
\end{algorithmic}

Now to explain some of the steps. The main step that I feel should be explained is the Master node: given that the inputs the master node recieves are source vectors, intuitively (I have no proof for this, in other words) it would seem that the unmixing matrices should just be permutation matrices.

Given that they are just permutation matrices and every site has its components ordered, sending the global W's back to each site would order the source vectors across sites in addition to within sites.

\section{Issues / Problems}




There are a couple of issues: 

\begin{enumerate}
    \item In theory, it would seem that all of the unmixing matrices should just be permutation matrices, since each site has the source vectors, they are just in the wrong order. We can eliminate scale differences by normalizing all the source vectors, however inital testing seems to say that this is not the case.

\end{enumerate}

\end{document}
